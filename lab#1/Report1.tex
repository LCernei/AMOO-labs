\documentclass[12pt,a4paper,titlepage]{article}
\usepackage{lab_style}
\usepackage{pdfpages}
\usepackage{eso-pic}

  
\begin{document}

\begin{titlepage}
\selectlanguage{english}

%----------------------------------------------------------------------------------------
% TITLE PAGE INFORMATION
%----------------------------------------------------------------------------------------
  \begin{center} % Center everything on the page

  %----------------------------------------------------------------------------------------
  % HEADING SECTIONS
  %----------------------------------------------------------------------------------------
  \textsc{\large Faculty of Computers, Informatics and Microelectronics}\\[0.5cm]
  \textsc{\large Technical University of Moldova}\\[1.2cm] % Name of your university/college
  \vspace{25 mm}

  \textsc{\Large Analysis of Object-Oriented Modeling}\\[0.5cm] % Major heading such as course name
  \textsc{\large Laboratory work \#1}\\[0.5cm] % Minor heading such as course title
  %\textsc{\large Laboratory work}\\[0.5cm] % Minor heading such as course title

\newcommand{\HRule}{\rule{\linewidth}{0.5mm}} % Defines a new command for the horizontal lines, change thickness here

  %----------------------------------------------------------------------------------------
  % TITLE SECTION
  %----------------------------------------------------------------------------------------
  \vspace{10 mm}
  \HRule \\[0.4cm]
  { \LARGE \bfseries Project description. User Story. Entities and Relations. }\\[0.4cm] % Title of your document
  \HRule \\[1.5cm]

  %----------------------------------------------------------------------------------------
  % AUTHOR SECTION
  %----------------------------------------------------------------------------------------
      \vspace{30mm}

      \begin{minipage}{0.4\textwidth}
      \begin{flushleft} \large
      \emph{Authors:}\\
      Cernei \textsc{Liviu}
      \end{flushleft}
      \end{minipage}
      ~
      \begin{minipage}{0.4\textwidth}
      \begin{flushright} \large
      \emph{Supervisor:} \\
      Mihail \textsc{Gavrilița} % Supervisor's Name
      \end{flushright}
      \end{minipage}\\[4cm]

      \vspace{5 mm}
      % If you don't want a supervisor, uncomment the two lines below and remove the section above
      %\Large \emph{Author:}\\
      %John \textsc{Smith}\\[3cm] % Your name

      %----------------------------------------------------------------------------------------
      % DATE SECTION
      %----------------------------------------------------------------------------------------

      %{\large \today}\\[3cm] % Date, change the \today to a set date if you want to be precise

      %----------------------------------------------------------------------------------------
      % LOGO SECTION
      %----------------------------------------------------------------------------------------

      %\includegraphics{red}\\[0.5cm] % Include a department/university logo - this will require the graphicx package

      %----------------------------------------------------------------------------------------

      \vfill % Fill the rest of the page with whitespace
      \end{center}
      
\end{titlepage}

\cleardoublepage

\newpage

\pagenumbering{arabic}
\setcounter{page}{1}
\setcounter{secnumdepth}{4}

\addtocontents{toc}{\protect\thispagestyle{empty}} % no page number on the table of contents page
\cleardoublepage


\phantomsection
\addcontentsline{toc}{section}{Introduction}
\section*{Laboratory work \#1}
\phantomsection

% \section{Purpose of the laboratory}
% Gain knowledge about basics of event-driven programming, understanding of window’s class and basic possibilities of Win32 API. Also she will try to understand and process OS messages.
\section{Tasks}
\begin{itemize}
	\item
	Describe your project (at least 1/2 A4 paper);
	\item 
	Give several user stories.
\end{itemize}

\section{Theory}
\subsection{Types of entities}
\begin{enumerate}  
	\item Structure
	\begin{enumerate}  
		\item Class
		\item Interface
		\item Use case
		\item Collaboration
		\item Active class
		\item Component
		\item Node 
	\end{enumerate}
	\item Behaviour
	\begin{enumerate}  
		\item Interaction
		\item Automate
	\end{enumerate}
	\item Grouping (package)
	\item Adnotation (note)
\end{enumerate}

\subsection{Types of relations}
\begin{enumerate}  
	\item Dependence
	\item Asociation
	\begin{enumerate}  
		\item Agregation
		\item Composition
	\end{enumerate}
	\item Generalization
	\item Realization 
\end{enumerate}

\section{Description of the project}
The new \textbf{St. Courses} platform is a great place to learn new skills and boost your knowledge. It provides a web based interaction between students and teachers. Because of it's simplicity and security, it can be used by schools and universities. On St. Courses, the user gets personalized help in order to customize his online classes.
The website is very user-friendly, engaging and easy to use. It also has a customizable front-end, advanced analytics and much more.

The platform is populated by 3 kinds of actors: administrators, teachers and students.

\textbf{Courses:}
A course has a title, description and content. It is placed in time by the start date and end date. It is created by a teacher and can have enroled students.
The content is structured in chapters and can contain text, images and interactive elements.
Every teacher who logged in can create and edit a course. To delete it, he must complete a request to an administrator, containing the cause of removal.

\textbf{Tests:}
The teacher optionally can create a test to evaluate the quality of the course. For this he must fill in the questions, provide the coorect answers and create an grading system.
For the student, a test is obligatory. At the end of the course the students get marks based on their activity.

\textbf{Feedback/Reporting:}
Both teachers and students can give feedback which wil be assigned to the course history. Also, they can report bad behaviour / inappropriate language / fake content, etc. Each reportmust be resolved by an administrator.

\textbf{Enrolment to a course:}
The courses are free, meaning that everyone with an account can enrol to any corse. However, there are recomended courses, based on age, fields of interest, difficulty level.
A student can lose access to a course if he unenroled himself, or the teacher removed him.
In both cases the student loses the progress, tests, marks gained in that course.

\textbf{Invitations:}
A teacher can send invitations to a course. The invited students will be enroled automatically. With this method the teacher can assemble an real-life grop and assure access to the course for them.

\subsection* {User stories:}
\begin{itemize}
	\item
	An administrator (moderator) is responsible for handling reports, removal of inapropriate content, offering technical support.
	\item 
	A teacher can create a new course, edit an existing one, approve or reject an student to a course, create tests.
	\item
	A student can enrol to a course, read the material, unenrol from a course, pass a test, give feedback.
\end{itemize}

\
\clearpage
\cleardoublepage

\end{document}