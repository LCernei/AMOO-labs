\documentclass[12pt,a4paper,titlepage]{article}
\usepackage{lab_style}
\usepackage{pdfpages}
\usepackage{eso-pic}

  
\begin{document}

\begin{titlepage}
\selectlanguage{english}

%----------------------------------------------------------------------------------------
% TITLE PAGE INFORMATION
%----------------------------------------------------------------------------------------
  \begin{center} % Center everything on the page

  %----------------------------------------------------------------------------------------
  % HEADING SECTIONS
  %----------------------------------------------------------------------------------------
  \textsc{\large Faculty of Computers, Informatics and Microelectronics}\\[0.5cm]
  \textsc{\large Technical University of Moldova}\\[1.2cm] % Name of your university/college
  \vspace{25 mm}

  \textsc{\Large }\\[0.5cm] % Major heading such as course name
  \textsc{\large Laboratory work \#X}\\[0.5cm] % Minor heading such as course title
  %\textsc{\large Laboratory work}\\[0.5cm] % Minor heading such as course title

\newcommand{\HRule}{\rule{\linewidth}{0.5mm}} % Defines a new command for the horizontal lines, change thickness here

  %----------------------------------------------------------------------------------------
  % TITLE SECTION
  %----------------------------------------------------------------------------------------
  \vspace{10 mm}
  \HRule \\[0.4cm]
  { \LARGE \bfseries Lab title  }\\[0.4cm] % Title of your document
  \HRule \\[1.5cm]

  %----------------------------------------------------------------------------------------
  % AUTHOR SECTION
  %----------------------------------------------------------------------------------------
      \vspace{30mm}

      \begin{minipage}{0.4\textwidth}
      \begin{flushleft} \large
      \emph{Authors:}\\
      Name \textsc{SURNAME}
      \end{flushleft}
      \end{minipage}
      ~
      \begin{minipage}{0.4\textwidth}
      \begin{flushright} \large
      \emph{Supervisor:} \\
      Mihai \textsc{Coșleț} % Supervisor's Name
      \end{flushright}
      \end{minipage}\\[4cm]

      \vspace{5 mm}
      % If you don't want a supervisor, uncomment the two lines below and remove the section above
      %\Large \emph{Author:}\\
      %John \textsc{Smith}\\[3cm] % Your name

      %----------------------------------------------------------------------------------------
      % DATE SECTION
      %----------------------------------------------------------------------------------------

      %{\large \today}\\[3cm] % Date, change the \today to a set date if you want to be precise

      %----------------------------------------------------------------------------------------
      % LOGO SECTION
      %----------------------------------------------------------------------------------------

      %\includegraphics{red}\\[0.5cm] % Include a department/university logo - this will require the graphicx package

      %----------------------------------------------------------------------------------------

      \vfill % Fill the rest of the page with whitespace
      \end{center}
      
\end{titlepage}

\cleardoublepage

\newpage

\pagenumbering{arabic}
\setcounter{page}{1}
\setcounter{secnumdepth}{4}

\addtocontents{toc}{\protect\thispagestyle{empty}} % no page number on the table of contents page
\cleardoublepage


\phantomsection
\addcontentsline{toc}{section}{Introduction}
\section*{Laboratory work \#X}
\phantomsection

\section{Purpose of the laboratory}
Gain knowledge about basics of event-driven programming, understanding of window’s class and basic possibilities of Win32 API. Also she will try to understand and process OS messages.
\section{Laboratory Work Requirements}
\begin{itemize}
	\item \textbf{Mandatory Objectives}
	      \begin{itemize}
		      \item Choose a \textit{Programming Style Guideline} that you'll follow
		      \item Create a \textbf{Windows application}
		      \item Add 2 \texttt{buttons} to window: one with default styles, one with custom styles (size, background, text color, font family, font size)
		      \item Add 2 \texttt{text} elements to window: one with default styles, one with custom styles (size, background, text color, font family, font size) \textit{[one of them should be something funny]}
		      \item On windows resize, one of the \texttt{text}s should "reflow" and be in window's center (vertically and horizontally)
	      \end{itemize}
	\item \textbf{Objectives With Points:}
	      \begin{itemize}
		      \item \texttt{(1pt)} Add 2 text inputs to window: one with default styles, one with custom styles (size, background, text color, font family, font size)
		      \item \texttt{(1pt)} Make elements to fit window on resize \textit{(hint: you can limit minimal window width and height)}
		      \item \texttt{(0-2pt)}Make elements to interact or change other elements (\texttt{1pt} each different interactions) \textit{(ex. on button click, change text element color or position)}
		      \item \texttt{(1pt)} Change behavior of different window actions (at least 3). For ex.: on clicking close button, move window to a random location on display's working space
		      \item \texttt{(1pt)} Write your own PSG (you can take existent one and modify it) and argue why it is better (for you)
	      \end{itemize}
\end{itemize}

\clearpage
\cleardoublepage

\section{Laboratory work implementation}

\subsection{Tasks and Points}

Here should be the list of the implemented tasks.

\subsection{Laboratory work analysis}

Add link to your repository.
Create a README.md file for each laboratory work you submit. It should include the tasks that you had been implemented.
Explain the features that you had been added to your window.

\subsection{Prove your work with screens}

Should be added 1 pic/screen for each implemented functionality.

\clearpage
\cleardoublepage

\phantomsection
\addcontentsline{toc}{section}{Conclusion}
\section*{Conclusion}
\phantomsection

Vivamus sed auctor quam, id posuere leo. Quisque sed posuere est. Phasellus ut ornare arcu. Mauris volutpat nunc arcu, quis porta ante blandit et. Nam eu nisl in ipsum auctor tristique at eget tortor. Mauris bibendum luctus turpis. Etiam sed urna ac purus pharetra maximus quis vitae tortor. Proin bibendum ante lorem, ut vestibulum tellus tempor in. Nam nec aliquam magna. Pellentesque aliquet scelerisque nunc, nec pretium dui ultrices at. Donec tempor sem vitae sapien bibendum rhoncus. Donec mollis, nulla non feugiat lobortis, justo tortor mattis enim, non placerat justo nibh vel dui. Maecenas nec justo et dui scelerisque eleifend nec nec elit. Phasellus ornare semper nibh et pellentesque. Etiam arcu nunc, tempor id ultrices eget, faucibus non neque. Donec ut sem quis dolor dictum elementum eu sit amet ipsum.

Aliquam sed sem eu massa tincidunt tincidunt. Praesent aliquam libero metus, et dictum mauris tincidunt vitae. Quisque interdum vulputate interdum. Suspendisse ut molestie erat. Aliquam a erat elementum odio blandit ornare vel at ante. Nam venenatis sagittis mi eget tempus. Pellentesque nibh ligula, suscipit vitae lorem et, vestibulum tincidunt nibh. In venenatis sollicitudin augue non fermentum. Ut sollicitudin eget nisl non cursus. Cras mattis nisi id justo consequat posuere. Duis gravida gravida tortor in aliquet.

\clearpage
\cleardoublepage

\cleardoublepage
% \addcontentsline{toc}{section}{References}
% \input{biblio.tex}
\cleardoublepage
   
\end{document}